\chapter{Sposoby rozwiązania}
\thispagestyle{chapterBeginStyle}
\label{rozdzial2}
\iffalse
\section{Dokładne rozwiązanie}
%Finding a minimal coloring can be done using brute-force search (Christofides 1971; Wilf 1984; Skiena 1990, p. 214)
\fi

\section{Programowanie dynamiczne}
Poniżej przedstawione algorytmy rozwiązują zagadnienie \textit{Minimalnego kolorowanie grafu},
metodą dynamiczną, które rozwiązują problem dokładnie. Pierwszym algorytmem jest rozwiązanie przedstawione przez 
Lawlera. Znajduje optymalne rozwiązanie w czasie $O(2.4423^n)$, gdzie $n$ to ilość wierzchołków\cite{LAWLER197666}.
Lepszą złożonością obliczeniową wykazał się algorytm Eppstein'a rozwiązuje on problem w czasie $O(2.4150^n)$ \cite{Epp-JGAA-03}.
Kolejnym krokiem wykazał się Byskov i osiągnął jak narazie najlepszy wynik dla \textit{Minimalnego Kolorowania Grafu} $O(2.4023^n)$ \cite{byskov2002chromatic}.


\section{Programowanie zachłanne}
Omówione poniżej algorytmy nie zapewniają optymalnego rozwiązania, często nie są jego bliskie, za to czas wykonywania jest znacznie szybszy.
Zaletą tego rodzaju algorytmów jest to, że mogą posłużyć do znalezienia górnego ograniczenia.

Algorytm \textit{DSATUR} zaproponowany przez Brélaz'a \cite{Brelaz}.
Alogorytm ten jest wciąż rozwijany oraz wykorzystywany, dlatego kod podstawowej wersji został umieszczony Algorithm \ref{DSATUR}, złożoność obliczeniowa wynosi $O(n^2)$.
Innym algorytmem zachłannym jest \textit{Recursive Largest First} (\textit{RLF}), zaprojektowanym przez \textit{Leighton} \cite{Lewis}. 
Algorytm działa w czasie $O(n^3)$, jest wolniejszy niż \textit{DSATUR}, a wyniki są porównywalne. 

\section{DSATUR-based branch and bound }
Metoda ta opiera się na przeszukiwaniu drzewa reprezentującego przestrzeń rozwiązań problemu. 
Stosowane w tej metodzie odcięcia redukują liczbę przeszukiwanych węzłów. 
\textit{DSATUR-based branch-and-bound} w literaturze najczęściej spotyka się skróconą wewrsje \textit{DSATUR}. 
W rozwiązaniu zasugewrował, aby rozpocząć kolorowanie od dużej kliki i etapu wstępnego przetwarzania przy użyciu heurystyki \textit{DSATUR}. 
\textit{Sewell} poprawił to podejście, wprowadzając nową strategię rozsądzania remisu \cite{Sewell1993AIA}.
\textit{Segundo} zasugerował zastosowanie tej strategii tylko selektywnie, dzięki temu poprawił ogólną wydajność.
Nazwał ten algorytm \textit{PASS} i został testowany na podzbiorze \textit{DIMACS}(\href{http://archive.dimacs.rutgers.edu/Challenges/}{informacje}) i na 
grafach losowych. Praca \textit{Fabio Furini, Virginie Gabrel, Ian-Christopher Ternier} polegała na zmodyfikowania podstawowej wersji by zmieniać dolną granice rozwiązania. 
Wyniki są lepsze dla grafów losowych o dużej gęstości $0.7 - 0.9$ \cite{furini}, na grafach \textit{DIMACS} uzyskuje podobne wyniki jak algorytm \textit{PASS}.
\\ \

\begin{algorithm}[H]
    \newenvironment{polishalgorithm}[1][]
    \KwData{graf G}
    \KwData{graf G}
    \KwResult{Niezależne zbiory $\mathbf{S}$}
    $\mathbf{S} \rightarrow \emptyset$ \\
    $X \rightarrow V(G) $ \\
    \While {$X \neq \emptyset$}{
        Wybierz $v \in X$\\
        \For{$j \leftarrow 1 do |\mathbf{S}|$}{
            \If{$S_j \cup \{v\}$ jest niezależny}{
                $S_j \leftarrow S_j \cup \{v\}$ \\
                $break$
            }
        }
        \If{$j > |\mathbf{S}|$}{
            $S_j \leftarrow \{v\}$ \\
            $\mathbf{S} \leftarrow \mathbf{S} \cup S_j $
        }
        $X \leftarrow X - \{v\}$
    }
    \caption{DSATUR}
    \label{DSATUR}
\end{algorithm}

\section{Programowanie liniowe}

\textit{Annuj Mehrotra, Michael A. Trick} zaprezentowali sposób rozwiązania problemu programowaniem liniowym \cite{Mehrotra}.
Metoda opiera się na rozwiązaniu problemu \textit{Maximum weighted independent set} wielokrotnie 
i znalezieniu rozwiązania całkowitoliczbowego, które jest odpowiednie do uzyskania rozwiązania bazowego problemu.

\textit{Gualandi, Stefano and Malucelli, Federico} rozwiązują problem łącząc wiele metod \cite{Gualandi}. 
Podejście generowania kolumn zostało ulepszone dzięki zastosowaniu programowania ograniczeń w celu 
rozwiązania podprogramu cenowego i obliczenia rozwiązań heurystycznych. Ponadto wprowadzili nowe techniki 
w celu poprawy wydajności generowania kolumn w rozwiązywaniu zarówno liniowej relaksacji,
jak i problemu liczby całkowitej. Dodatkowo rozszerzyli swoje rozwiązania do wyliczenia problemu \textit{Minimum Vertex Graph Multicoloring}.









\iffalse


\subsection{Algorytm Lawlera}



\begin{algorithm}[H]
    \newenvironment{polishalgorithm}[1][]
    \KwData{graf G}
    \KwData{graf G}
    \KwResult{Liczba chromantyczna}
    $n \leftarrow := |V(G)|$ \\
    $X \leftarrow $ tablica od $0$ do $2^n - 1$ \\
    $X[0] \leftarrow 0$ \\
    \For{$S\leftarrow 1$ \KwTo $2^n - 1$}{
        $s \leftarrow f(S)$ \\
        $X[s] \leftarrow \inf$ \\
        \For{$I\leftarrow  1$ \KwTo $2^n - 1$}{

        }
    }
        
        \caption{LARAC}
\end{algorithm}

\section{Metoda Lawlera} 
Lawler przedstawił 



\section{Metody heurystyczne}
%Brelaz's heuristic algorithm can be used to find a good, but not necessarily minimum vertex coloring.


\fi