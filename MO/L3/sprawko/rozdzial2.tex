\chapter{Zadanie 2}
\thispagestyle{chapterBeginStyle}
\label{rozdzial2}

\section{Opis problemu}
Zadanie polega na zminimalizowaniu kosztu wykonania zadań $J = \{1,2,...,n\}$.
Każde zadanie posiada czas wykoniania $p_{j}$, wagę zadania $w_{j}$ 
oraz moment moment gotowości (moment przed który zadanie nie może być wykonane) $r_j$.
Funkcja kosztu: 
$\sum_{\substack{j \in J}} w_j C_j$,
gdzie $C_j$ jest czasem zakończenia zadania $j$. 

\section{Rozwiązanie}
Do rozwiązania problemu stworzono program z parametrami: 
\begin{itemize}
    \item $p_{j}$ gdzie $j \in J$ opisuje czas wykonania zadania;
    \item $w_{j}$ gdzie $j \in J$ opisuje wagę zadania;
    \item $r_{j}$ gdzie $j \in J$ opisuje moment dostępności zadania;
\end{itemize}
\   \\
Model rozwiązujący posiada:
\begin{itemize}
    \item Zmienną $x_{j,h}$ gdzie $j \in J, h \in Czas$ opisującą czy zadanie $j$ uruchomi się w momencie $h$;
    \item Ogarniczenie $\forall \substack{j \in J} \sum_{\substack{h \in Czas}} x[j,h] == 1$ ogranicza dokładnie jedno uruchomienie każdego zadania;
    \item Ogarniczenie $\forall \substack{j \in J} \sum_{\substack{h \in Czas}} x[j,h] >= r_j$ ogranicza moment dostępności zadania;
    \item Ogarniczenie $\forall \substack{h \in Czas} \sum_{\substack{j \in J}} x[j,h] <= 1$ ogranicza nakładanie się zadań w tym samym czasie.
\end{itemize}

\section{Wyniki i interpretacja}
W celu sprawdzenia rozwiązania uruchomiono program z parametrami podanych w tabeli \ref{tabela_zad2_dane}. 
\begin{table}[ht]
    \begin{center}
        \begin{tabular}{| c | c | c | c |} 
            \hline
            \rowcolor{lgray}
            Zadanie & Czas & dostępność & waga \\ [0.5ex] 
            \hline
            1 & 1 & 5 & 50\\
            \hline
            2 & 2 & 0 & 1 \\
            \hline
            3 & 3 & 0 & 2 \\
            \hline
            4 & 4 & 0 & 3 \\
            \hline
            5 & 5 & 0 & 5 \\
            \hline
        \end{tabular}
        \caption{Przykładowe dane}
        \label{tabela_zad2_dane}
    \end{center}
\end{table}
\   \\
\   \\
Koszt rozwiązania dla danego problemu wyniósł 414, co zaprezentowane jest w tabeli \ref{tabela_zad2_rozwiazanie}.


\begin{table}[ht]
    \begin{center}
        \begin{tabular}{| c | c |} 
            \hline
            \rowcolor{lgray}
            Czas & Zadanie \\ [0.5ex] 
            \hline
            \hline
            1 & 4 \\
            \hline
            5 & 1 \\
            \hline
            6  & 5 \\
            \hline
            11 & 3 \\
            \hline
            14 & 2 \\
            \hline
        \end{tabular}
        \caption{Rozwiązanie}
        \label{tabela_zad2_rozwiazanie}
    \end{center}
\end{table}


