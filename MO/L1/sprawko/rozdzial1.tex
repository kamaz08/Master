\chapter{Zadanie 1}
\thispagestyle{chapterBeginStyle}
\label{rozdzial1}

\section{Opis problemu}
Zadanie polega na przetestowaniu algorytmów $LP$ pod względem dokładności i odporności
 rozwiązująć następujące zagadnienie

    $$min \ c^T x$$
przy warunkach
    $$Ax=b, \ x \geq 0,$$
gdzie
 $$a_{i,j} = \frac{1}{i+j-1}, i,j = 1,2,..,n$$
 $$c_{i}=b_{i}=\sum^{n}_{j=1} \frac{1}{i+j-1}, i,j=1,2,...,n$$

 \section{Rozwiązanie}
 Do rozwiązania zadania napisano program obliczający równanie, jako dane wejściowe przyjmuje
rozmiar $n$ macierzy $A$. Po skończeniu pracy program wypisuje na standardowe wyjście błąd względny. 

\section{Wyniki i interpretacja}

Uruchamiając program dla danych wejściowych $n = 2,3,...,10$ otrzymano błędy względne zawarte w tabeli \ref{tabela_zad1}.

\begin{table}[ht]
    \begin{center}
        \begin{tabular}{|c | c|} 
            \hline
            \rowcolor{lgray}
            n & błąd względny \\ [0.5ex] 
            \hline
            2 & 1.05325004057301e-15  \\ 
            \hline
            3 & 3.67157765110227e-15 \\
            \hline
            4 & 3.27016385075681e-13 \\
            \hline
            5 & 3.35139916635905e-12 \\
            \hline
            6 & 6.83335790676898e-11 \\
            \hline
            7 & 1.67868542192291e-08 \\
            \hline
            8 & 0.514058972177268 \\
            \hline
            9 & 0.682911338087722 \\
            \hline
            10 & 0.990387574803086 \\
            \hline
        \end{tabular}
        \caption{Błedy względne macierzy}
        \label{tabela_zad1}
    \end{center}
\end{table}

Dla danych wejściowych $2,3,...,7$ wyniki są akceptowalne, jednak dla większego rozmiaru macierzy
błąd względny jest za duży by rozwiązać ten układ równań. Przyczyną takiego zachowania jest uwarunkowanie macierzy
Hilberta, który wynosi 
$cond(H_n) = O\left({\frac{e^{3.5255n}}{\sqrt(n)}}\right)$ dla $n$ oznaczającego rozmiar macierzy.


