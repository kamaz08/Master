\chapter{Zadanie 1}
\thispagestyle{chapterBeginStyle}
\label{rozdzial1}

\section{Opis problemu}
Zadanie polega na zminimalizowaniu ilości odpadów w tartaku. Firma otrzymuje zamówienie na deski
różnej szerokości, które są wycinane z desek o standardowej szerokości. Odpadem nazywane są deski,
które przekraczają ilość zapotrzebowania oraz takie które nie mają odpowiedniego rozmiaru. 
Danymi do zadania są: 
\begin{itemize}
    \item rozmiar deski standardowej (22 cali),
    \item zamówienia na deski o rozmiarach 7, 5, 3 cali(110 o szerkości 7 cali, 120 o szerokości 5 cali oraz 80 o szerkości 3 cali).
\end{itemize}

 \section{Rozwiązanie}
 Do rozwiązania zadania napisano program z parametrami:
 \begin{itemize}
    \item szerokość standardowej deski
    \item zapotrzebowanie na deski o szerokości 3 cali
    \item zapotrzebowanie na deski o szerokości 5 cali
    \item zapotrzebowanie na deski o szerokości 7 cali
\end{itemize}
Program wyznacza możliwe sposoby podzielenia jednej deski na 3,5,7 calowe deski oraz odpad. Następnie
minimalizuje odpad. 

\section{Wyniki i interpretacja}
Plan podziału desek przedstawiony jest w tabeli \ref{tabela_zad1}. Łączna suma nadmiaru to 18 cali. 

\begin{table}[ht]
    \begin{center}
        \begin{tabular}{|c | c| c| c| c|} 
            \hline
            \rowcolor{lgray}
            ilość & 7cal & 5cal & 3cal & odpad \\
            \hline
            9 & 1 & 0 & 5 & 0\\
            \hline
            28 & 1 & 3 & 0 & 0\\
            \hline
            37 & 2 & 1 & 1 & 0\\
            \hline
            \rowcolor{lgray}
            suma & 111 & 121 & 82 & 0\\
            \hline
        \end{tabular}
        \caption{Podział desek}
        \label{tabela_zad1}
    \end{center}
\end{table}
